\documentclass[usenatbib,usegraphicx,letterpaper]{mn2e}
\usepackage[totalwidth=480pt,totalheight=680pt]{geometry}

\usepackage{amssymb}
\usepackage{epsfig}
\usepackage{amsmath}
\usepackage{color}
%\usepackage{hyperref}

\bibliographystyle{mn2e} 

%-------- journals
\newcommand{\araa}{ARAA~}
\newcommand{\apj}{ApJ~}
\newcommand{\apjl}{ApJL~}
\newcommand{\apjs}{ApJS~}
\newcommand{\mnras}{MNRAS~}
\newcommand{\nat}{Nature~}
\newcommand{\physrep}{Phys. Rep.~}
\newcommand{\aj}{AJ~}

%%%% Misc %%%
\newcommand{\beq}{\begin{equation}}
\newcommand{\eeq}{\end{equation}}
\newcommand{\beqray}{\begin{eqnarray}}
\newcommand{\eeqray}{\end{eqnarray}}

\newcommand{\ben}{\begin{enumerate}}
\newcommand{\een}{\end{enumerate}}
\newcommand{\bit}{\begin{itemize}}
\newcommand{\eit}{\end{itemize}}

%%%%%%%%  galaxy properties  %%%%%%%%
\newcommand{\rhalf}{R_{1/2}}
\newcommand{\rhalfdisk}{R_{1/2}^{\rm disk}}
\newcommand{\rhalfbulge}{R_{1/2}^{\rm bulge}}
\newcommand{\adisk}{A_{\rm disk}}
\newcommand{\abulge}{A_{\rm bulge}}
\newcommand{\alphadisk}{\alpha_{\rm disk}}
\newcommand{\alphabulge}{\alpha_{\rm bulge}}
\newcommand{\rvir}{R_{\rm vir}}
\newcommand{\bt}{{\rm B/T}}
\newcommand{\mstar}{M_{\ast}}

%%%%%%%%  halo properties  %%%%%%%%
\newcommand{\halospin}{\lambda_{\rm halo}}
\newcommand{\mvir}{M_{\rm vir}}
\newcommand{\macc}{M_{\rm acc}}


%%%%%%%%  observations  %%%%%%%%
\newcommand{\rproj}{r_{\rm p}}
\newcommand{\wproj}{w_{\rm p}}


%%%%%%%%%%%%%%%%%%%%%%%%%%%%%%%%
%%%%%%%%%%%%%%%%%%%%%%%%%%%%%%%%


\usepackage{epsfig}  \usepackage{graphicx}   \usepackage{rotating}

\begin{document}

\title[The Galaxy Size--Halo Connection]
{On the Galaxy Size--Halo Connection}


\author[Hearin, Behroozi, Kravtsov \& Moster]{
Andrew Hearin$^{1}$, Peter Behroozi$^{2}$, Andrey Kravtsov$^{3}$, Benjamin Moster$^{4}$\\
$^{1}$Argonne National Laboratory, Argonne, IL, USA 60439, USA\\
$^{2}$Department of Physics, University of Arizona, 1118 E 4th St, Tucson, AZ 85721 USA\\
$^{3}$Department of Astronomy \& Astrophysics, The University of Chicago, Chicago, IL 60637 USA\\
$^{4}$Universit{\"a}ts-Sternwarte, Ludwig-Maximilians-Universit{\"a}t M{\"u}nchen, Scheinerstr. 1, 81679 M{\"u}nchen, Germany
}

\maketitle

\begin{abstract}
We derive empirical modeling constraints on the connection between dark matter halos and the half-mass radius $\rhalf$ of galaxy bulges and disks. We show that both $\rhalfdisk$ and $\rhalfbulge$ are well-described by power law scaling relations with halo virial radius, $\rhalf=A\rvir^{\alpha}.$ Novel to this work, we use new SDSS measurements of the $\rhalf-$dependence of galaxy clustering to constrain the model parameters, $A_{\rm bulge}, A_{\rm disk}, \alpha_{\rm bulge},\alpha_{\rm disk},$ and log-normal scatter $\sigma_{R_{1/2}}.$ Even when only coarsely tuning these parameters to the observed one-point functions $\langle\rhalfdisk|\mstar^{\rm disk}\rangle$ and $\langle\rhalfbulge|\mstar^{\rm bulge}\rangle,$ our model accurately predicts the observed two-point clustering on small- and large-scales. This success non-trivial, as we show that galaxy clustering is highly sensitive to the physics that shapes satellite galaxy profiles. We find no evidence for the commonly assumed relation between halo spin $\halospin$ and $\rhalfdisk,$ and show that this assumption cannot be meaningfully constrained with either the clustering or lensing of  $L_{\ast}$ galaxies. Our results provide simple boundary conditions for more complex and fine-grained models of galaxy size. We make our python code publicly available to support cosmological surveys that require realistic synthetic galaxy populations.
\end{abstract}

\section{Introduction}
\label{sec:intro}
Some introduction goes here.

\section*{Acknowledgments}

\end{document}







