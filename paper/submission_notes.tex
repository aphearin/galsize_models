Clustering Constraints on the Relative Sizes of Central and Satellite Galaxies

Andrew Hearin, Peter Behroozi, Andrey Kravtsov, Benjamin Moster

We empirically constrain how galaxy size relates to halo virial radius using new measurements of the size- and stellar mass-dependent clustering of galaxies in the Sloan Digital Sky Survey.  We find that small galaxies cluster much more strongly than large galaxies of the same stellar mass. The magnitude of this clustering difference increases on small scales, and decreases with increasing stellar mass. Using forward modeling techniques implemented in Halotools, we test an empirical model in which present-day galaxy size is proportional to the size of the virial radius at the time the halo reached its maximum mass. This simple model reproduces the observed size-dependence of galaxy clustering in striking detail. The success of this model provides strong support for the conclusion that satellite galaxies have smaller sizes relative to central galaxies of the same halo mass. Our findings indicate that satellite size is set prior to the time of infall, and that a remarkably simple, linear size--virial radius relation emerges from the complex physics regulating galaxy size. We make quantitative predictions for future measurements of galaxy-galaxy lensing, including dependence upon size, scale, and stellar mass, and provide a scaling relation of the ratio of mean sizes of satellites and central galaxies as a function of their halo mass that can be used to calibrate hydrodynamical simulations and semi-analytic models.


12 pages plus an appendix. Submitted to MNRAS.
Figure 5 shows that a simple empirical model, with R50 = 0.01Rvir, can accurately reproduce new measurements of size-dependent clustering of SDSS galaxies. Figure 9 shows predictions for the size-dependence of future lensing measurements. Figure 10 provides a diagnostic for hydro sims and SAMs



